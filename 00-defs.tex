% Copyright (C) 2023  Andrea Patrizi (AndrePatri, andreapatrizi1b6e6@gmail.com)
% 
% This file is part of PhDBiorobBeamerTemplate and distributed under the General Public License version 2 license.
% 
% PhDBiorobBeamerTemplate is free software: you can redistribute it and/or modify
% it under the terms of the GNU General Public License as published by
% the Free Software Foundation, either version 2 of the License, or
% (at your option) any later version.
% 
% PhDBiorobBeamerTemplate is distributed in the hope that it will be useful,
% but WITHOUT ANY WARRANTY; without even the implied warranty of
% MERCHANTABILITY or FITNESS FOR A PARTICULAR PURPOSE.  See the
% GNU General Public License for more details.
% 
% You should have received a copy of the GNU General Public License
% along with PhDBiorobBeamerTemplate.  If not, see <http://www.gnu.org/licenses/>.
% 
% Command to extract the first frame
% Assumes extract.bash is in the same directory and -shell-escape is enabled
%\newcommand{\extractframe}{
%    \immediate\write18{./extract.bash}
%}

% Define a command to show or hide slides with an argument
\newcommand{\showHiddenSlides}[1]{%
	\ifnum#1=1
	\newcommand{\hiddenSlide}[1]{##1} % Show content
	\else
	\ifnum#1=0
	\newcommand{\hiddenSlide}[1]{}  % Hide content
	\else
	\errmessage{Invalid argument for \noexpand\showHiddenSlides: #1}
	\fi
	\fi
}

% Define a new environment named tight_itemize
\newenvironment{tight_itemize}{
	\begin{itemize}[topsep=1pt, partopsep=1pt, parsep=1pt, itemsep=1pt]
	}{
	\end{itemize}
}

% Define a new command named \custombox
% #1: alignment
% #2: width of the box
% #3: color
% #4: text to be displayed inside the box
\newcommand{\custombox}[4]{
	\begin{tcolorbox}[width=#2, colback=#3, colframe=black, left=2pt, right=2pt, top=2pt, bottom=2pt,halign=#1, valign=center]
		#4
	\end{tcolorbox}
}

\forestset{
	customtree/.style={
		for tree={ % apply settings to all nodes in the tree
%			draw, % draw a border around the node
%			fill=blue!20, % fill the node with a light blue color
			rounded corners, % round the corners of the nodes
			edge={very thick, -latex}, % use thick arrows for edges
			l sep+=0.1cm, % increase level distance
		}
	}
}

% Custom command to create title page
\NewDocumentCommand{\titleslide}{m o o}{
	\setbeamertemplate{footline}[nofootline]
	\begin{frame}
		\centering
		\includegraphics[width=0.85\paperwidth]{#1} % Include logo/image
		\\[0.5cm]
		\large{\getYear~year~PhD~annual~evaluation}\\[0.3cm]
		\Large{\getProjName}\\[0.5cm]
		
		\begin{columns}[c, onlytextwidth]
			\column{0.2\textwidth}
			\centering
			\IfValueTF{#2}{ 
				\includegraphics[height=1.0\textwidth]{#2}
			}{}
		
			\column{0.6\textwidth}
			\centering
			\large{Student:~\getStudentName}\\[0.3cm] % Display author
			\small{Tutors:~\getTutorsList}\\[0.8cm] % Display author
			\small{\getLocationName, \insertdate} % Display date
			\vfill
			
			\column{0.2\textwidth}
			\centering
			\IfValueTF{#3}{ 
				\includegraphics[height=1.0\textwidth]{#3}
			}{}
		\end{columns}
	\end{frame}
	\setbeamertemplate{footline}[customfootline]
}

\newcommand{\closingslide}[1]{
	\setbeamertemplate{footline}[nofootline]
	\begin{frame}
		\centering
		\includegraphics[width=0.85\paperwidth]{#1} % Include logo/image
			         	\\[0.5cm]
			         	\vfill
			         	\Large{Thanks for your attention!}\\[0.3cm]
			         	\Large{Questions?}\\[0.8cm]
		\normalsize{Contacts: \getContactsList}
		\vfill
	\end{frame}
	\setbeamertemplate{footline}[customfootline]
}

% Wrapper command around \includemedia
% #1: width
% #2: height
% #3: video filename (without extension)
% #4: video file extension (optional, defaults to mp4)
\newcommand{\includevideo}[4][mp4]{
    \includemedia[
        width=#2,
        height=#3,
        activate=pageopen,
        deactivate=pageclose,
        addresource=videos/#4.#1,
        flashvars={
            &source=videos/#4.#1
            &autoPlay=true
            &loop=true
            &scale=tofit
        }
    ]{\includegraphics[width=#2, height=#3]{videos/extracted_frames/#4_first_frame.jpg}}{VPlayer9.swf}
}

%\NewDocumentCommand{\includevideo}{m m m O{mp4}}{
%    \centering
%    \includemedia[
%        width=#2,
%        height=#3,
%        activate=pageopen,
%        deactivate=pageclose,
%        addresource=videos/#1.#4,
%        flashvars={
%            &source=videos/#1.#4
%            &autoPlay=true
%            &loop=true
%            &scale=tofit
%        }
%    ]{\includegraphics[width=#2, height=#3]{videos/extracted_frames/#1_first_frame.jpg}}{VPlayer9.swf}
%}

\setlist[itemize]{itemsep=4pt,parsep=2pt,topsep=2pt,leftmargin=0.5cm,labelsep=4pt}

% colors
\definecolor{lightgray}{RGB}{180,180,180}
\definecolor{barblue}{RGB}{153,204,254}
\definecolor{groupblue}{RGB}{51,102,254}
\definecolor{linkred}{RGB}{165,0,33}
\definecolor{linkgreen}{RGB}{20,182,117}
\definecolor{linkorange}{RGB}{205,134,41}
\definecolor{green}{RGB}{0, 128, 0}
\definecolor{red}{RGB}{255, 0, 0}
\definecolor{lightblue}{RGB}{173, 216, 230}
\definecolor{darklightblue}{RGB}{121, 182, 201}
\definecolor{mydarkblue}{RGB}{0, 0, 139}
\definecolor{mydarkgreen}{RGB}{0, 100, 0}
\definecolor{mylightgreen}{RGB}{56, 93, 75}
\definecolor{boxbackgroundgreen}{RGB}{191, 216, 193}
\definecolor{boxbackgroundblue}{RGB}{174, 220, 243}
\definecolor{boxbackgroundgrey}{RGB}{201, 212, 218}
\definecolor{titlebackgroundblue}{RGB}{105, 171, 222}
\definecolor{titlebackgroundblue2}{RGB}{142, 186, 220}
\definecolor{boxbackgroundyellow}{RGB}{225, 213, 154}
\definecolor{boxbackgroundlightlightblue}{RGB}{167, 242, 239}
\definecolor{boxbackgroundlightlighttorquoise}{RGB}{196, 189, 242}
\definecolor{boxbackgroundlightlightred}{RGB}{249, 183, 200}

\setganttlinklabel{s-s}{START-TO-START}
\setganttlinklabel{f-s}{FINISH-TO-START}
\setganttlinklabel{f-f}{FINISH-TO-FINISH}
\sffamily

% cmd for full citations
\newlength{\mylength} % define a new length \mylength
\newcommand{\fullcitelist}[3][1em]{%
	\setlength{\mylength}{\leftmargin+#1} % set \mylength to \leftmargin + #1
	\begin{enumerate}[label={[#2.\arabic*]}, labelsep=1em, leftmargin=\mylength]
		\forcsvlist{\item \fullcite}{#3}
	\end{enumerate}%
}

\ExplSyntaxOn
\seq_new:N \g_tutorlist_seq
\seq_new:N \g_contactslist_seq

\NewDocumentCommand{\setCurriculumName}{m}
{
	\tl_gset:Nn \g_curriculum_tl {#1}
}

\NewDocumentCommand{\setFootlineLogo}{m}
{
	\tl_gset:Nn \g_footline_logo_tl {#1}
}

\NewDocumentCommand{\setLocation}{m}
{
	\tl_gset:Nn \g_location_tl {#1}
}

\NewDocumentCommand{\setResearchProjName}{m}
{
	\tl_gset:Nn \g_proj_tl {#1}
}

\NewDocumentCommand{\setTutors}{m}
{
	\seq_gset_from_clist:Nn \g_tutorlist_seq {#1}
}

\NewDocumentCommand{\setStudent}{m}
{
	\tl_gset:Nn \g_student_tl {#1}
}

\NewDocumentCommand{\setCycle}{m}
{
	\tl_gset:Nn \g_cycle_tl {#1}
}

\NewDocumentCommand{\setYear}{m}
{
	\tl_gset:Nn \g_year_tl {#1}
}

\NewDocumentCommand{\setContacts}{m}
{
	\seq_gset_from_clist:Nn \g_contactslist_seq {#1}
}

\NewDocumentCommand{\getCurriculumName}{}{\tl_use:N \g_curriculum_tl}

\NewDocumentCommand{\getFootlineLogo}{}{\tl_use:N \g_footline_logo_tl}

\NewDocumentCommand{\getLocationName}{}{\tl_use:N \g_location_tl}

\NewDocumentCommand{\getProjName}{}{\tl_use:N \g_proj_tl}

\NewDocumentCommand{\getStudentName}{}{\tl_use:N \g_student_tl}

\NewDocumentCommand{\getTutorsList}{}
{
	\seq_use:Nn \g_tutorlist_seq {,~}
}

\NewDocumentCommand{\getCycle}{}
{
	\tl_use:N \g_cycle_tl
}

\NewDocumentCommand{\getYear}{}
{
	\tl_use:N \g_year_tl
}

\NewDocumentCommand{\getContactsList}{}
{
	\seq_use:Nn \g_contactslist_seq {,~}
}

\NewDocumentCommand{\checksetup}{}
{
	\seq_if_empty:NT \g_tutorlist_seq
	{
		\msg_error:nn {PhdBiorobReportTemplate} {no-tutors-set}
	}
	\seq_if_empty:NT \g_contactslist_seq
	{
		\msg_error:nn {PhdBiorobReportTemplate} {no-contacts-set}
	}
	\tl_if_empty:NT \g_student_tl
	{
		\msg_error:nn {PhdBiorobReportTemplate} {no-student-set}
	}
	\tl_if_empty:NT \g_proj_tl
	{
		\msg_error:nn {PhdBiorobReportTemplate} {no-projectname-set}
	}
	\tl_if_empty:NT \g_curriculum_tl
	{
		\msg_error:nn {PhdBiorobReportTemplate} {no-curriculum-set}
	}
	\tl_if_empty:NT \g_cycle_tl
	{
		\msg_error:nn {PhdBiorobReportTemplate} {no-cycle-set}
	}
	\tl_if_empty:NT \g_year_tl
	{
		\msg_error:nn {PhdBiorobReportTemplate} {no-year-set}
	}
}

\msg_new:nnn {PhdBiorobReportTemplate} {no-tutors-set} {No~tutors~have~been~set!~Please~use~\space\string\setTutors.}

\msg_new:nnn {PhdBiorobReportTemplate} {no-contacts-set} {No~contacts~set!~Please~use~\space\string\setContacts\space.}

\msg_new:nnn {PhdBiorobReportTemplate} {no-student-set} { No~student~has~been~set!~Please~use~\string\setStudent.}

\msg_new:nnn {PhdBiorobReportTemplate} {no-curriculum-set} { No~curriculum~has~been~set!~Please~use~\string\setCurriculumName.}

\msg_new:nnn {PhdBiorobReportTemplate} {no-projectname-set} { No~research~project~name~has~been~set!~Please~use~\string\setResearchProjName.}

\msg_new:nnn {PhdBiorobReportTemplate} {no-cycle-set} { No~PhD~cycle~has~been~set!~Please~use~\space\string\setCycle.}

\msg_new:nnn {PhdBiorobReportTemplate} {no-year-set} { No~academic~year~has~been~set!~Please~use~\space\string\setYear.}

\NewDocumentCommand{\signature}{}
{
	\vfill
	\noindent
	\begin{minipage}{\textwidth}
		\begin{flushright}
			\seq_map_inline:Nn \g_tutorlist_seq
			{
				##1 \\ \rule{60mm}{.55pt} \\[1ex]
			}
		\end{flushright}
	\end{minipage}
}

\NewDocumentCommand{\addPersonalData}{}
{
	\begin{flushleft}
		\vspace{1.5cm}
		\large{\textbf{Student:~}}\large{{\getStudentName}}\vspace{0.2cm}\\
		\large{\textbf{Cycle:~}}\large{\getCycle}\vspace{0.2cm}\\
		\large{\textbf{Tutor(s):~}}\large{{\getTutorsList}}\vspace{0.2cm}\\
		\large{\textbf{Year:~}}\large{\getYear}\vspace{0.2cm}\\
		\large{\textbf{Contacts:~}}\large{\getContactsList}
	\end{flushleft}
}

\NewDocumentCommand{\addResearchProjTitle}{}
{
	\begin{center}
		\textbf{\normalsize{\getProjName}}
	\end{center}
}

\NewDocumentCommand{\addTopLogos}{}
{
	\begin{figure}[t]
		\centering
		\includegraphics[width=0.99\textwidth]{logos/top_logos.pdf}
	\end{figure}
	\hphantom{h}\vspace{1.3cm}
}

\NewDocumentCommand{\addTitlePage}{}
{
	\addTopLogos
	
	\begin{center}
		
		\LARGE{\textbf{Doctorate~in~Bioengineering~and~Robotics }\vspace{0.4cm}\\
			Curriculum:~\getCurriculumName}
		
	\end{center}
	
	\addPersonalData % adds personal data to frontpage
	
	\clearpage
}

\ExplSyntaxOff

\newcolumntype{P}[1]{>{\raggedright\arraybackslash}p{#1}}
